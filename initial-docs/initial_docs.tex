\documentclass{article}

\usepackage{geometry}
\usepackage{amsmath}
\usepackage{graphicx}
\usepackage[super]{nth}
\usepackage[T1]{fontenc}

\title{Analiza wydźwięku w języku angielskim na bazie recenzji produktów -- projekt NLP2022Z}
\author{Olga Sapiechowska\\Maciej Marcinkiewicz}
\date{12 grudnia 2022}

\newgeometry{lmargin=3.2cm, rmargin=3.2cm, bmargin=2.5cm}

\begin{document}

\maketitle

\section{Wstęp}

\subsection{Temat projektu}

Projekt polega na pobraniu opinii w języku angielskim o produktach z kategorii: proszki do prania kolorów, tabletki do zmywarki i kubki termiczne oraz zrealizowania modelu wykrywającego wydźwięk: pozytywny, neutralny, negatywny. Składa się z następujących zadań do zrealizowania:
\begin{enumerate}
    \item Pobranie opinii z portali internetowych i przygotowanie korpusu.
    \item Stworzenie modelu.
    \item Stworzenie drugiego modelu wykorzystującego dane dostępne w literaturze w ramach dotrenowania wykorzystywanego pretrenowanego modelu, np. BERT.
\end{enumerate}

\subsection{Definicja problemu}

\section{Metody analizy sentymentów}

\section{Opis proponowanego rozwiązania}

\subsection{Pobieranie i przygotowywanie danych}

\subsection{Modele wykrywające wydźwięk}

\section{Projekt implementacji}


% \begin{figure}[h!] %possible: b, t, h, p and override (!)
%     \centering
%         \includegraphics[width=0.8\linewidth]{path-to-picture.png}
%     \caption{Sample text}
% \end{figure}

\end{document}
